%\nonstopmode
\hbadness=100000
\documentclass[a4paper, 12pt]{article}
\usepackage{verbatim,amsmath,graphicx,geometry,textcomp,url,caption}
\geometry{ a4paper, total={170mm,257mm}, left=20mm, top=20mm}

\usepackage[toc, page]{appendix}
\usepackage[dvipsnames]{xcolor}
\definecolor{subr}{rgb}{0.8, 0.33, 0.0}
\definecolor{func}{rgb}{0.76, 0.6, 0.42}

\begin{document}
\begin{center}
MCNP Report										\\
Ocean Wong (Hoi Yeung Wong)						\\
MSc Physics and Technology of Nuclear Reactors	\\
2019-03-09 										\\
\end{center}

\section{Exercise 1 Simple Neutron Source in a Bucket of Water}
\subsection{Input file}
Energy cut-off is not applied so that very slow (thermalized) neutrons to interact and let further reactions take place.\\
Therefore the lower bound of the $1^{st}$ bin is 0.\\
The temperature of the cross-section data in databse 42(ENDL92, acquired by Lawrence Livermore National Laboratory) were acrquired at T=300K, as shown in 91-99, and is subsequently adjusted down to 20.4 $C^{o}$ ($2.53 \times 10^{-8}$MeV). Either way, thermal effects should significantly affect that falls into the first 3 energy bins. ($0$-$10^{-9}$MeV, $10^{-9}$-$10^{-8}$MeV,$10^{-8}$-$10^{-7}$MeV respectively). The energy group structure was not chosen to be finer because of
\\*Insert the pictures of the cross sections of the geometry here, caption with cell number and material number\\

\subsection{Output file}
By examining the first 50 particles(using \texttt{PRINT 110}), the source was confirmed to be a point souce 2cm above the centre of the bottom of the tank's internal surface; and the majority of the particles have initial energy $E<4$ MeV as expected when they are distributed according to the Watt spectrum for neutron generated by ${}^{235}U$+n(thermal)\\
\\*Insert Watt Spectrum .png\\
\\*plot variation as number of particles increases up to 20000\\
\begin{enumerate}
	\item These results are not reliable because the statistical tests (\\*insert number of them not passed\\) are not passed, meaning that some reactions are not sampled enough for us to be confidence about the frequencies of their occurance.
	\item The total fluences $\Phi$ are simplay calculated by formula $\sum_i \Phi_i$ 
		\\where each $\Phi_i$ refers to the time-integrated flux calculated for a single energy bin.
		\\      A = Effective area for surface flux tallying, where the particle passing through still had non-zero weight.
\end{enumerate}
\section{Exercise 3: Criticality}
\subsection{Questions}
\begin{itemize}
	\item Examine and report upon the estimate of $k_{eff}$ with cycle number given in the output.
	\item Are you confident in the final reported result and its uncertainty? Justify your answer.
	\begin{itemize}
		\item make sure that the 10 statistical checks pass on all cells
	\end{itemize}
	\item In addition to Monte Carlo stochastic uncertainties what other uncertainties may need 
	to be considered in a criticality safety assessment?
\end{itemize}

\end{document}